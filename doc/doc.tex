% Authors:
%
% Andrej Pavlovič <xpavlo14@vutbr.cz>
% Tverdokhlib Vladyslav V. <xtverd01@vutbr.cz>


\documentclass[a4paper, 11pt]{article}


\usepackage[czech]{babel}
\usepackage{graphicx}
\graphicspath{{pictures/}}
\DeclareGraphicsExtensions{.pdf,.png,.jpg}
\usepackage[utf8]{inputenc}
\usepackage[left=2cm, top=3cm, text={17cm, 24cm}]{geometry}
\usepackage{times}
\usepackage{graphicx}
\usepackage[hyphens]{url}
\usepackage[unicode, colorlinks, hypertexnames=false]{hyperref}
\usepackage[czech, boxed]{algorithm2e}
\usepackage[perpage]{footmisc}
\usepackage{indentfirst}
\hypersetup{
    colorlinks=true,
    linkcolor=black,
    urlcolor=black,
}

\begin{document}
	%%%%%%%%%%%%%%%%%%%%%%%%%%%%%%%% Title page %%%%%%%%%%%%%%%%%%%%%%%%%%%
	\begin{titlepage}
		\begin{center}
			\includegraphics[width=0.8 \linewidth]{FIT_logo.pdf}

			\vspace{\stretch{0.382}}

			\Huge{Simulační studie} \\
			\LARGE{\textbf{Prevoditelnost plánovanýh emisií aut v EU do roku 2050}} \\
			\Large{Tým: AnderTeam} \\
			\Large{EU Green Deal a jeho dopady na EU ekonomiku}

			\vspace{\stretch{0.618}}
		\end{center}

		\begin{minipage}{0.5 \textwidth}
			\Large
			12. 12. 2021
		\end{minipage}
		\hfill
		\begin{minipage}[r]{0.5 \textwidth}
			\Large
			\begin{tabular}{ll}
			    Andrej Pavlovič & (xpavlo14) \\
				Tverdokhlib Vladyslav V. & (xtverd01)
			\end{tabular}
		\end{minipage}
	\end{titlepage}
	\clearpage

	%%%%%%%%%%%%%%%%%%%%%%%%%%%%%%%% Contents %%%%%%%%%%%%%%%%%%%%%%%%%%%%%%%%%%%%%
	\pagenumbering{arabic}
	\setcounter{page}{1}
	\tableofcontents
	\clearpage

	%%%%%%%%%%%%%%%%%%%%%%%%%%%%%%%% 1. Introduction %%%%%%%%%%%%%%%%%%%%%%%%%%%%%%%%%%%%%%

	\section{Úvod}
    Tato práce je v rámci předmětu IMS na FIT VUT. Cílem bylo prostudovat pakt Green Deal, jeho dopad na různé aspekty státu a modelovat proveditelnost ve vybraných zemích.
    Zaměřili jsme se na plán na snížení emisí CO2 od osobních automobilů v Evropě. Provedli jsme simulační studie k předpovědi proveditelnosti či neproveditelnosti a porovnali s jinými plány snižování emisí. Model se táhne od roku 2020 do roku 2050.

	\subsection{Autori, zdroje}
	
	Projekt vypracovali studenti Andrej Pavlovič a Tverdokhlib Vladyslav V.
    
    Data byla shromážděna z různých článků, studií, statistik a byla kombinována. Odkazy na materiály jsou v bibliografii. Zdroje byly použity výhradně z internetu. Zaměřili jsme se na rok 2020 jako na skutečný rok, protože Green Deal vznikl v lednu 2019.
	\subsection{Ověření validity}
	Platnost modelu byla zkontrolována za běhu. Ověření bylo především prostřednictvím prohlášením plánu Green Deal.

	%%%%%%%%%%%%%%%% 2. Analysis of the topic and used methods / technologies %%%%%%%%%%%%%%%
	\section{Rozbor témy a~použitých metod/technologií}
    Existuje Paris agreement, která vstoupila v platnost v roce 2016 a která zejména stanoví opatření ke snížení emisí CO2 od nových osobních aut: 15\% do roku 2025, 37.5\% do roku 2030.
    
    Green Deal je plán přijatý v roce 2019, který zavazuje členy EU učinit evropský kontinent klimaticky neutrálním k 2050 roku. Navíc navrhuje zpřísnit snížení emisí nových osobních aut o 55\% do roku 2030 a dokonce o 100\% do roku 2035.
	\subsection{Fakty}
    V souladu s Regulation (EU) 2019/631, byl zaveden vzorec pro stanovení cíle emisí CO2 pro jednoho výrobce aut pro daný rok (SPECIFIC EMISSIONS TARGETS FOR PASSENGER CARS).
    V podstatě jsou v něm konstanty a mění se v závislosti na roce. Jediné, co potřebujeme získat, je \textit{průměrná hmotnost vozu z konkrétního vozového parku za konkretní rok}. Právě výpočty podle tohoto vzorce bude nutné dodržet, aby bylo dosaženo snížení emisí každého vozového parku výrobců. Bude také potřebná \textit{průměrná ujetá vzdálenost za rok}, pro zobrazení objemu co2 v gramech.

    V druhé části modelu budeme vycházet z tendence vývoje automobilového parku konkrétní země. Pak potřebujeme následující údaje: \textit{celkový počet automobilů v zemi}, \textit{objem výroby automobilů ročně} v zemi a \textit{\% snížení vozového parku země každý rok} (7\%).
	\subsection{Popis použitých postupů/metod/technologií}
	Když nacházíme \textit{průměrnou hmotnost vozu} a \textit{průměrnou ujetou vzdálenost} použijeme generátor náhodných čísel z rozsahu (od minimální po maximální hodnotu za 10 let podle statistik).
	
	Také jsme si vypůjčili určitý koeficient pro vzorec s 95\% pravdivostí.

	%%%%%%%%%%%%%%%%%%%%%%%%%%%%%%%% 3. Concept of model %%%%%%%%%%%%%%%%%%%%%%%%%%%
	\vspace*{14mm}
	\section{Koncepce modelu}

	\subsection{Popis konceptuálneho modelu}
	% \label{section:conceptual_model_desc}
	PART ONE:
    
    \vspace*{3mm}2021, 2025, 2030, 2035 - milestony proveditelnosti. V 1. části máme model implementace v konkrétních milestonech.
    Díky vzorci budeme počítat Emission Target pro konkrétní zemi na konkrétní rok. 
    
    \vspace*{3mm}
    
    \begin{itemize}
    \item Emissions Target = T + a × (M - M0) g/km
    
    kde:
    \vspace*{2mm}

	T = 95 g/km for 2021. 
	
	- 80,75 g/km for 2025. 
	
	- 42,75 g/km for 2030 (59.4 g/km in Paris agreement).
	
	- 0 g/km for 2035.
	
	\vspace*{2mm}
	
    a = 0.0333 for 2020-2024. 
    
    - 0.0214 for 2025-2030 (95\% forecast).
    
    \vspace*{2mm}
    
    M0 = 1 379,88 kg for 2020-2021
    
    - 1 398,50 kg for 2022-2030
    
    \vspace*{2mm}
    
    M(parameter) = average mass of the cars kg.
    \end{itemize}
    
    \vspace*{3mm}
    
    Na základě této hodnoty bude vypočítáno \textit{minimální snížení emisí} v \% a \textit{maximální možné emise CO2 na jednoho ridice} pro každý milestone vzhledem k roku 2020 (nová auta).
    
    \vspace*{3mm}
    
    \begin{itemize}
    \item Min. reduction = (1 - (ET / E2020)) * 100 \%
    
    kde:
    \vspace*{2mm}
    
    ET = emissions target g/km
    
    E2020 = emissions in 2020 g/km
    
    \vspace*{3mm}
    
    \item Max. CO2 emissions for one car = ET * distance / 1000
        
    kde:
    \vspace*{2mm}
    
    ET = emissions target g/km
    
    distance = travelled distance km
    \end{itemize}
    
    \vspace*{3mm}
    
    Také uvidíme, jak se sníží Emission Target.
    
    \vspace*{3mm}
    
    \begin{itemize}
    \item ET reduction = (1 - (ET / start\_ET)) * 100 \%
        
    kde:
    \vspace*{2mm}
    
    ET = emissions target g/km
    
    start\_ET = ET in 2020 g/km
    \end{itemize}
    
    \vspace*{6mm}
    
    PART TWO:
    
    \vspace*{3mm}
    
    V tomto jsme se pokusili simulovat každoroční snižování objemu emisí, pokud bude nastaven growth rate podílu elektromobilů (hodnota v \%, o kterou se v každém příštím roce by zvýšil podíl elektromobilů).
    
        \vspace*{3mm}
    
    for(i=1;i\leq30;i++)
    \{
    
        \vspace*{1mm}
    
        \quad if(year from 2021 to 2024)
        
        \vspace*{1mm}
        
            \quad \quad ET(2021)
            
        \vspace*{1mm}
        
        \quad if(year from 2025 to 2029)
        
        \vspace*{1mm}
        
            \quad \quad ET(2025)
            
        \vspace*{1mm}
        
        \quad if(year from 2030 to 2034)
        
        \vspace*{1mm}
        
            \quad \quad ET(2030)
            
        \vspace*{1mm}
        
        \quad if(year from 2035)
        
        \vspace*{1mm}
        
            \quad \quad ET(2035)
            
        \vspace*{1mm}
      \}

        ec\_share = ec\_growth*i;
        
        \vspace*{1mm}
        
        if(ec\_share gt 1)\{
        
        \vspace*{1mm}
        
            \quad ec\_share = 1;    
        
        \vspace*{1mm}
        \}

        total\_old\_gascars = total\_old\_gascars * parkreduct;
        
        \vspace*{1mm}
        
        total\_registered\_gascars += (1-ec\_share) * registered\_cars;
        
        \vspace*{1mm}
    
        \quad print ( i + 2020: 1 - (((total\_old\_gascars * start\_ET) + (total\_registered\_gascars * ET)) / (total\_gascars * start\_ET)) );
        
        \vspace*{1mm}
        
        
        
    kde:
    \vspace*{3mm}
    
     total\_old\_gascars = total cars in a country.
     
     \vspace*{2mm}
     
     total\_registered\_gascars = new gasoline cars registered.
     
     \vspace*{2mm}
     
     registered\_cars = number of registered cars yearly. 
     
     \vspace*{2mm}
     
     ec\_share = share of electric cars of registered cars yearly.
     
     \vspace*{2mm}
     
     parkreduct = rate of reduction.

	%%%%%%%%%%%%%%%%%%% 4. Simulation model architecture %%%%%%%%%%%%%%%%%%%%%%%%
	\section{Architektura simulačního modelu/simulátoru}
	% \label{section:simulation_model_architecture}
	
	Model byl napsán v jazyce C++, standard C++11. Byl použit kompilátor g++

	%%%%%%%%%%%%%%% 5. The principle of simulation experiments and their course %%%%%%%%%%%
	\section{Podstata simulačních experimentů a~jejich průběh}
	
	\subsection{Experimenty}
	Experimenty boli rozdelené do dvoch častí. Experiment číslo 1 patrí do prvej časti, experimenty 2 a 3 patria do druhej. %TODO <-
	
	\subsubsection{Experiment 1}
	
    Na základě tohoto experimentu jsme zdůraznili rozdíl v Pařížské dohodě a Zelené dohodě. 
    
    Viděli jsme, jak se procento snížení u těchto dvou plánů mění.
    
    Také obdržel statistické údaje - maximální emise na nové auto v konkrétním roce. \\ 
    
	Spouštel se s parametrami: -n 'avgE', 'minD', 'maxD', 'minM', 'maxM'
	
	    \begin{itemize}
            \item emise 2020 pro Franci
            \item minimální ujetá vzdálenost
            \item maximální ujetá vzdálenost
            \item minimální masa vozdila
            \item maximální masa vozdila
        \end{itemize}
    v tomto pořadí. \\
    
	Spuštení: \\
	
	\texttt{./source -n 98.47 12223 13802 1293 1337} \\
	
	Výsledky (Pre každú tabuľku najprv green deal [HORE], potom Paris agreement [DOLE]):
	
	\begin{center}
        \begin{tabular}{|c|c|c|}
        \hline
         Rok & Min. reduction \% & Max. CO2 emissions kg (from new cars) \\ 
         \hline
         2021 & 5,37\% & 1166 \\  
         \hline
         2025 & 19,59\% & 990,9 \\
         \hline
         2030 & 58,18\% & 515,3 \\  
         \hline\hline
         2030 & 41,3\% & 723,7 \\  
         \hline
         2035 & 100\% & 0 \\
         \hline
        \end{tabular}
    \end{center}
    
    \begin{center}
        \begin{tabular}{|c|c|c|}
        \hline
         Rok & ET & ET reduction \% (new cars) \\ 
         \hline
         2021 & 93,17 & 0\% \\  
         \hline
         2025 & 79,18 & 15,02\% \\
         \hline
         2030 & 41,18 & 55,81\% \\  
         \hline\hline
         2030 & 57,83 & 37,94\% \\  
         \hline
         2035 & 0 & 100\% \\
         \hline
        \end{tabular}
    \end{center}
    
    \subsubsection{Experiment 2}
	
    V tomto experimentu pozorujeme nárůst procenta snížení oproti roku 2020
    
    Spouštel se s parametrami: -g 'growth', 'gascars', 'regcars', 'parkreduct', 'minM', 'maxM'
    
	Spuštení: ./source -g 0.1 31380000 1700000 0.93 1293 1337
	
	\includegraphics[width=\textwidth]{2exp.PNG}
	
	\subsection{Závěry experimentů}

	%%%%%%%%%%%%%%%%% Summary of simulation experiments and conclusion %%%%%%%%%%%%%%%%%%
	\section{Zhrnutie simulačných experimentov a~záver}
	Na závěr můžeme potvrdit realizaci plánu pomocí vzorce pro výpočet Emission Target ve Francii.
	
	\vspace*{5mm}
	
    Vidíme, že rozdíl mezi Pařížskou dohodou a Grean Deal není velký, a to nemá velký vliv na proveditelnost.
    
    \vspace*{5mm}
    
    Podle posledního diagramu vidíme, že pokud zavedeme pouze elektromobily rovnoměrně, tak do roku 2050 budeme mít 88\% snížení emisí CO2 oproti roku 2020.
	
                                                    

	
	
	
	%%%%%%%%%%%%%%%%% Bibliography %%%%%%%%%%%%%%%%%%

	\clearpage
	\nocite{*}
	\renewcommand{\refname}{Literatura}
	\bibliography{bibliography}
	\bibliographystyle{czechiso}
\end{document}
